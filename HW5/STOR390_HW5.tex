% Options for packages loaded elsewhere
\PassOptionsToPackage{unicode}{hyperref}
\PassOptionsToPackage{hyphens}{url}
%
\documentclass[
]{article}
\usepackage{amsmath,amssymb}
\usepackage{iftex}
\ifPDFTeX
  \usepackage[T1]{fontenc}
  \usepackage[utf8]{inputenc}
  \usepackage{textcomp} % provide euro and other symbols
\else % if luatex or xetex
  \usepackage{unicode-math} % this also loads fontspec
  \defaultfontfeatures{Scale=MatchLowercase}
  \defaultfontfeatures[\rmfamily]{Ligatures=TeX,Scale=1}
\fi
\usepackage{lmodern}
\ifPDFTeX\else
  % xetex/luatex font selection
\fi
% Use upquote if available, for straight quotes in verbatim environments
\IfFileExists{upquote.sty}{\usepackage{upquote}}{}
\IfFileExists{microtype.sty}{% use microtype if available
  \usepackage[]{microtype}
  \UseMicrotypeSet[protrusion]{basicmath} % disable protrusion for tt fonts
}{}
\makeatletter
\@ifundefined{KOMAClassName}{% if non-KOMA class
  \IfFileExists{parskip.sty}{%
    \usepackage{parskip}
  }{% else
    \setlength{\parindent}{0pt}
    \setlength{\parskip}{6pt plus 2pt minus 1pt}}
}{% if KOMA class
  \KOMAoptions{parskip=half}}
\makeatother
\usepackage{xcolor}
\usepackage[margin=1in]{geometry}
\usepackage{graphicx}
\makeatletter
\def\maxwidth{\ifdim\Gin@nat@width>\linewidth\linewidth\else\Gin@nat@width\fi}
\def\maxheight{\ifdim\Gin@nat@height>\textheight\textheight\else\Gin@nat@height\fi}
\makeatother
% Scale images if necessary, so that they will not overflow the page
% margins by default, and it is still possible to overwrite the defaults
% using explicit options in \includegraphics[width, height, ...]{}
\setkeys{Gin}{width=\maxwidth,height=\maxheight,keepaspectratio}
% Set default figure placement to htbp
\makeatletter
\def\fps@figure{htbp}
\makeatother
\setlength{\emergencystretch}{3em} % prevent overfull lines
\providecommand{\tightlist}{%
  \setlength{\itemsep}{0pt}\setlength{\parskip}{0pt}}
\setcounter{secnumdepth}{-\maxdimen} % remove section numbering
\ifLuaTeX
  \usepackage{selnolig}  % disable illegal ligatures
\fi
\IfFileExists{bookmark.sty}{\usepackage{bookmark}}{\usepackage{hyperref}}
\IfFileExists{xurl.sty}{\usepackage{xurl}}{} % add URL line breaks if available
\urlstyle{same}
\hypersetup{
  pdftitle={HW 5},
  pdfauthor={Tianrui Ye},
  hidelinks,
  pdfcreator={LaTeX via pandoc}}

\title{HW 5}
\author{Tianrui Ye}
\date{12/29/2023}

\begin{document}
\maketitle

This homework is meant to give you practice in creating and defending a
position with both statistical and philosophical evidence. We have now
extensively talked about the COMPAS \footnote{\url{https://www.propublica.org/datastore/dataset/compas-recidivism-risk-score-data-and-analysis}}
data set, the flaws in applying it but also its potential upside if its
shortcomings can be overlooked. We have also spent time in class
verbally assessing positions both for an against applying this data set
in real life. In no more than two pages \footnote{knit to a pdf to
  ensure page count} take the persona of a statistical consultant
advising a judge as to whether they should include the results of the
COMPAS algorithm in their decision making process for granting parole.
First clearly articulate your position (whether the algorithm should be
used or not) and then defend said position using both statistical and
philosophical evidence. Your paper will be grade both on the merits of
its persuasive appeal but also the applicability of the statistical and
philosohpical evidence cited.

\emph{STUDENT RESPONSE}

To the Honorable Judge:

As a statistical consultant entrusted with advising on the integration
of the COMPAS recidivism risk score data into the decision-making
process for granting parole, it is imperative to approach this
responsibility with a comprehensive understanding of both the
statistical underpinnings and the profound ethical implications of the
tool in question. After careful consideration, it is my recommendation
that reliance on the COMPAS system for parole decisions be reconsidered
due to significant concerns regarding its accuracy, inherent biases, and
the philosophical implications of its application.

\textbf{Statistical Evidence Against the Use of COMPAS} The COMPAS
recidivism risk score, while innovative in its attempt to bring
data-driven insights into the judicial process, has shown considerable
limitations in its predictive accuracy and fairness. Investigations,
such as the extensive analysis conducted by ProPublica, reveal troubling
disparities in the algorithm's performance across different racial
groups. Specifically, the system was found to erroneously label black
defendants as likely to reoffend at a rate nearly twice that of white
defendants, raising serious concerns about the perpetuation of
historical biases encoded within the data it analyzes.

Furthermore, the proprietary nature of the COMPAS algorithm
significantly hinders any form of independent verification or scrutiny.
This opacity is at odds with the principles of transparency and
accountability essential to the justice system. Without the ability to
review and understand the mechanisms behind its decisions, the fairness
and reliability of the COMPAS system remain in question.

\textbf{Philosophical and Ethical Concerns} Beyond the statistical
shortcomings, the use of COMPAS raises fundamental ethical questions
about justice, fairness, and the role of technology in the judicial
system. At the heart of these concerns is the principle of individual
justice --- the idea that each case deserves consideration of its unique
circumstances, beyond what any algorithm can quantify. Relying on COMPAS
to inform parole decisions risks undermining this principle, treating
individuals as mere data points and neglecting the complex
socio-economic factors that contribute to criminal behavior.

Moreover, the reliance on such technology risks further entrenching
systemic biases, given that algorithms can only learn from the data they
are given. If this data reflects historical inequalities, the
algorithm's predictions will likely perpetuate these injustices. This
not only affects the individuals unjustly labeled by the system but also
erodes public trust in the fairness and impartiality of the legal
process.

\textbf{Counterarguments and Rebuttals} Proponents of COMPAS argue that
it introduces a level of consistency and objectivity to parole
decisions, potentially reducing human error or bias. While these are
valid considerations, they do not sufficiently address the fundamental
issues raised by the system's inaccuracies and biases. The quest for
efficiency and consistency should not come at the cost of fairness and
justice. Human judgment, with all its imperfections, allows for empathy,
understanding, and the possibility of redemption --- qualities that an
algorithm cannot replicate.

\textbf{Conclusion} In conclusion, while the COMPAS system presents an
appealing vision of a data-driven judicial process, its current
implementation is fraught with issues that compromise both the integrity
of parole decisions and the broader principles of justice and fairness.
As such, I strongly advise against its use in determining parole
eligibility. Instead, I urge a renewed focus on enhancing the
transparency, accountability, and humanity of the parole decision
process, investing in alternative methods that support rehabilitative
efforts and address the root causes of recidivism. It is only through
such measures that we can hope to achieve a more equitable and just
criminal justice system.

\end{document}
